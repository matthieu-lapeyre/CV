\documentclass[10pt,a4paper]{moderncv}
\moderncvtheme[blue]{classic}
\usepackage[utf8]{inputenc}
\usepackage{url}
\usepackage[top=3cm, bottom=1.5cm, left=2.5cm, right=2.5cm]{geometry}

\firstname{Matthieu}
\familyname{Lapeyre}
\title{Ingenieur de recherche}
\address{16 rue du Jardin Public}{33000 Bordeaux}
\email{matthieulapeyre@gmail.com}
\homepage{www.linkedin.com/in/matthieulapeyre}
\mobile{06 03 39 98 51}
\extrainfo{29/06/1987 -- Permis B}
% \photo[75pt][0pt]{photo}
\begin{document}
\maketitle

\section{Expériences} % (fold)
\cventry{Juin 2012 -- Aujourd'hui}{Projet Poppy}{INRIA Bordeaux, Flowers Team}{}{}{Conception et réalisation d'un robot humanoide open-source et imprimé en 3D. Voir \url{http://www.poppy-project.org} pour plus d'information.}
\cventry{20014 -- Aujourd'hui}{Ingénieur de recherche}{INRIA Bordeaux, Flowers Team}{}{}{}
\cventry{2010 -- 2014}{Thèse}{INRIA Bordeaux, Flowers Team}{}{}{}
% \subsection{Stages} % (fold)
\cventry{2010}{Stage Master2}{INRIA Bordeaux, Flowers Team}{6 mois}{}{Etude de la marche bipède robotique passive et semi-passive.}
\cventry{2009}{Stage Master1}{CEA Cadarache (IRFM/SIPP/GIPM)}{6 mois}{}{Conception et intégration d'un dispositif d'analyses et de récupération des poussières dans les réacteurs de fusion nucléaire (projet ITER).}
% \cventry{2008}{Mathématiques}{Cours Particuliers}{}{}{niveau Collège/Lycée}
% \cventry{2006/2007}{Manutentionnaire}{E.T.L.S Lapeyre (Transport\&Logistique)}{1 mois}{}{}
% subsection stages (end)
\section{Compétences}
\subsection{Robotique}
\cvcomputer{Conception}{Product design, Mecanismes et Transmission, Integration éléctronique, Capteurs et Actuation}{CAO}{Solidworks, Catia}
\cvcomputer{Prototypage}{Impression 3D, Arduino}{Simulation}{SW Simulation, Matlab, Ansys}

\subsection{Outils Informatiques}
\cvcomputer{Édition}{Suite MS Office, \LaTeX, Google Docs}{Présentation}{PowerPoint, Prezi, Keynote, OmniGraffle}
\cvcomputer{OS}{Windows XP/7/8, Mac 0SX, Linux}{Photo/video}{Photoshop, iMovie}

\subsection{Informatique}
\cvcomputer{Script}{Matlab, Python}{Web}{HTML, CSS}
\cvcomputer{Autres}{Arduino, Jekyll, C++, Java, LUA, Fortran}{Versioning}{Git, SVN}

\subsection{Langues}
\cvcomputer{Anglais}{lu, parlé, écrit}{}{}

\section{Formation} % (fold)
\cventry{2010}{Diplôme de l'École Normale Supérieure de Cachan}{Ecole Normale Supérieure de Cachan}{}{}{}
\cventry{2008--2010}{Master TACS}{Ecole Normale Supérieure de Cachan}{}{Mention Bien}{Techniques Avancées en Calcul des Structures}
\cventry{2007--2008}{Licence SMTI}{Ecole Normale Supérieure de Cachan \& Université Pierre et Marie Curie}{}{}{Science Mécaniques et Techniques de l'Ingenieur}
\cventry{2005--2007}{Licence L1-L2}{Université Pierre et Marie Curie (Paris VI)}{}{}{Mécanique}
\cventry{2005}{Baccalauréat}{Lycée Theillard de Chardin (94)}{}{}{ Série Scientifique, spé physique}


\section{Centres d'intérêt} % (fold)
\cvline{Sports}{Tennis, Jogging, Football}
\cvline{Loisirs}{Bricolage, Musiques, Jeux video, Robotique, Voyages...}
% \cvline{Associatif}{Ancien président, vice-président de l’association VachemenRock. Organisation des Festivals 2007 et 2008 (direction, communication, installations techniques)}


\newpage
\nocite{*}
\bibliographystyle{unsrt}
\bibliography{publications.bib}

\end{document}
